\section{Отношения (бинарное, n-арное). Связь множеств и упорядоченной пары. Прямое произведение множеств. Композиция отношений. Степень отношения. Ядро отношения. Примеры.}

\subsection*{Упорядоченные пары и наборы}

\paragraph{Упорядоченная пара}
Для объектов $a$ и $b$ \textbf{упорядоченная пара} обозначается $(a, b)$.
\textbf{Равенство} упорядоченных пар:
$$(a, b) = (c, d) \stackrel{\text{def}}{\iff} a = c \land b = d.$$
В общем случае $(a, b) \ne (b, a)$.

\paragraph{Упорядоченный набор ($n$-ка, кортеж)}
Упорядоченный набор из $n$ элементов обозначается $(a_1, \ldots, a_n)$. Набор может быть определен рекурсивно: $(a_1, \ldots, a_n) \stackrel{\text{def}}{=} ((a_1, \ldots, a_{n-1}), a_n)$.
\textbf{Длина} набора: $|(a_1, \ldots, a_n)| = n$.

\textbf{Теорема о равенстве наборов:} Два набора одной длины равны, если равны их соответствующие элементы:
$$\forall n \ge 1: (a_1, \ldots, a_n) = (b_1, \ldots, b_n) \iff \forall i \in \{1, \ldots, n\} (a_i = b_i).$$

\subsection*{Прямое произведение множеств}

\paragraph{Прямое (декартово) произведение двух множеств}
Для множеств $A$ и $B$ \textbf{прямым произведением} $A \times B$ называется множество всех упорядоченных пар, в которых первый элемент принадлежит $A$, а второй принадлежит $B$:
$$A \times B \stackrel{\text{def}}{=} \{(a, b) \mid a \in A \land b \in B\}.$$

\textbf{Теорема о мощности:} Для конечных множеств $A$ и $B$ мощность произведения равна произведению мощностей: $|A \times B| = |A| \cdot |B|$.

\paragraph{$n$-кратное прямое произведение}
Прямое произведение $n$ множеств $A_1, \ldots, A_n$ --- это множество наборов (кортежей):
$$A_1 \times \cdots \times A_n \stackrel{\text{def}}{=} \{(a_1, \ldots, a_n) \mid a_1 \in A_1 \land \cdots \land a_n \in A_n\}.$$

\paragraph{Степень множества}
\textbf{Степенью} множества $A$ называется его $n$-кратное прямое произведение самого на себя:
$$A^n \stackrel{\text{def}}{=} \underbrace{A \times \cdots \times A}_{n \text{ раз}}.$$
\textbf{Следствие:} $|A^n| = |A|^n$.

\subsection*{Отношения (Бинарное и $n$-арное)}

\paragraph{Бинарное отношение}
\textbf{Бинарным отношением} $\mathcal{R}$ между множествами $A$ и $B$ называется тройка $(A, B, R)$, где $R$ --- \textbf{подмножество прямого произведения} $A \times B$:
$$R \subset A \times B.$$
\begin{itemize}
    \item $R$ называется \textbf{графиком отношения}.
    \item $A$ --- \textbf{область отправления} (домен), $B$ --- \textbf{область прибытия} (кодомен).
    \item Если $A = B$ ($R \subset A^2$), то $\mathcal{R}$ называется \textbf{отношением на множестве} $A$.
\end{itemize}
Для бинарных отношений используется \textbf{инфиксная форма записи}: $a \mathcal{R} b \stackrel{\text{def}}{\iff} (a, b) \in R$.

\paragraph{Характеристики бинарного отношения}
Пусть $R \subset A \times B$.
\begin{itemize}
    \item \textbf{Область определения} ($\text{Dom } R$): Множество элементов $A$, участвующих в парах отношения:
    $$\text{Dom } R \stackrel{\text{def}}{=} \{a \in A \mid \exists b \in B ((a, b) \in R)\}.$$
    \item \textbf{Область значений} ($\text{Im } R$): Множество элементов $B$, участвующих в парах отношения:
    $$\text{Im } R \stackrel{\text{def}}{=} \{b \in B \mid \exists a \in A ((a, b) \in R)\}.$$
    \item \textbf{Обратное отношение} ($R^{-1} \subset B \times A$):
    $$R^{-1} \stackrel{\text{def}}{=} \{(b, a) \mid (a, b) \in R\}.$$
    \item \textbf{Дополнение отношения} ($\overline{R} \subset A \times B$):
    $$\overline{R} \stackrel{\text{def}}{=} \{(a, b) \mid (a, b) \notin R\} = (A \times B) \setminus R.$$
\end{itemize}

\paragraph{$n$-арное отношение}
\textbf{$n$-местное} ($n$-арное) отношение $\mathcal{R}$ --- это подмножество прямого произведения $n$ множеств:
$$R \subset A_1 \times A_2 \times \cdots \times A_n \stackrel{\text{def}}{=} \{(a_1, \ldots, a_n) \mid a_i \in A_i\}.$$

\subsection*{Композиция отношений и Степень}

\paragraph{Композиция отношений}
Пусть $R_1 \subset A \times C$ и $R_2 \subset C \times B$. \textbf{Композицией} отношений $R_1$ и $R_2$ называется отношение $R = R_1 \circ R_2 \subset A \times B$, определяемое:
$$a (R_1 \circ R_2) b \stackrel{\text{def}}{\iff} \exists c \in C (a R_1 c \land c R_2 b).$$

\textbf{Теорема:} Композиция отношений \textbf{ассоциативна}:
$$\forall R_1 \subset A \times B, R_2 \subset B \times C, R_3 \subset C \times D: (R_1 \circ R_2) \circ R_3 = R_1 \circ (R_2 \circ R_3).$$
\textbf{ЗАМЕЧАНИЕ:} Композиция отношений, в общем случае, \textbf{не коммутативна} ($R_1 \circ R_2 \ne R_2 \circ R_1$).

\paragraph{Степень отношения}
Пусть $R$ --- отношение на множестве $A$ ($R \subset A^2$). \textbf{Степенью} отношения $R$ называется его $n$-кратная композиция с самим собой:
$$R^n \stackrel{\text{def}}{=} \underbrace{R \circ R \circ \cdots \circ R}_{n \text{ раз}}.$$
По определению: $R^0 \stackrel{\text{def}}{=} I$ (тождественное отношение), $R^1 \stackrel{\text{def}}{=} R$, и $R^n \stackrel{\text{def}}{=} R^{n-1} \circ R$.

\subsection*{Ядро отношения}

\paragraph{Определение Ядра}
Пусть $R \subset A \times B$ --- отношение между множествами $A$ и $B$.
\textbf{Ядром отношения} $R$ называется \textbf{композиция} отношения $R$ с его обратным отношением $R^{-1}$:
$$\ker R \stackrel{\text{def}}{=} R \circ R^{-1} \subset A^2.$$
Другими словами, $a_1$ находится в отношении $\ker R$ с $a_2$ тогда и только тогда, когда существует хотя бы один общий элемент $b \in B$, связанный с обоими:
$$a_1 \ker R a_2 \stackrel{\text{def}}{\iff} \exists b \in B (a_1 R b \land a_2 R b).$$
Ядро отношения $R$ между $A$ и $B$ является \textbf{отношением на $A$}: $R \subset A \times B \implies \ker R \subset A^2$.

\paragraph{Свойства Ядра (Теорема)}
Ядро любого отношения рефлексивно и симметрично на своей области определения ($\text{Dom } R$):
\begin{enumerate}
    \item \textbf{Рефлексивность:} $\forall a \in \text{Dom } R (a \ker R a)$.
    \item \textbf{Симметричность:} $\forall a, b \in \text{Dom } R (a \ker R b \implies b \ker R a)$.
\end{enumerate}

\paragraph{Примеры}
\begin{enumerate}
    \item Пусть $R \subset \text{U} \times \mathbf{2^{\text{U}}}$ --- отношение принадлежности $\in$. Ядро отношения $\in$ универсально: $\ker (\in) = \text{U} \times \text{U}$.
    \item Отношение $N_1$ на $\mathbb{Z}$: $N_1 = \{(n, m) \mid |n - m| \le 1\}$.
    Ядром этого отношения является отношение $N_2$ (находиться на расстоянии не более 2):
    $$\ker N_1 = N_1 \circ N_1^{-1} = N_2 = \{(n, m) \mid |n - m| \le 2\}.$$
    \item Ядром отношения «тесного включения» ($X \subset Y \iff X \subset Y \land |X|+1=|Y|$) является отношение «отличаться не более чем одним элементом»:
    $$\ker (\subset_1) = \{(X, Y) \in \mathbf{2^{\text{U}}} \times \mathbf{2^{\text{U}}} \mid |X| = |Y| \land |X \cap Y| \ge |X| - 1\}.$$
\end{enumerate}
\textbf{ЗАМЕЧАНИЕ:} Ядро отношения $\ker R$ всегда является \textbf{отношением толерантности} на $\text{Dom } R$, так как оно рефлексивно и симметрично.