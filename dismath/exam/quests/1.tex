\section{Множества. Задание множеств. Парадокс Рассела.}

\subsection*{Множества}

При построении доступной для рационального анализа картины мира часто ис-
пользуется термин \textbf{«объект»} для обозначения некой сущности, отделимой от
остальных. Выделение объектов --- это не более чем произвольный акт нашего
сознания.

\paragraph{Элементы и множества}
Понятие \textbf{множества} принадлежит к числу фундаментальных понятий матема-
тики. Можно сказать, что множество --- это любая определённая \textbf{совокупность
объектов}. Объекты, из которых составлено множество, называются его \textbf{элемента-
ми}. Элементы множества различны и отличимы друг от друга.

Если объект $x$ является элементом множества $M$, то говорят, что $x$ принадлежит $M$.
\textbf{Обозначение:} $x \in M$.
В противном случае говорят, что $x$ не принадлежит $M$.
\textbf{Обозначение:} $x \notin M$.

\paragraph{Примеры}
\begin{itemize}
    \item Множество $S$ страниц в данной книге.
    \item Множество $\mathbb{N}$ натуральных чисел $\{1, 2, 3, \ldots\}$.
    \item Множество $P$ простых чисел $\{2, 3, 5, 7, 11, \ldots\}$.
    \item Множество $\mathbb{Z}$ целых чисел $\{\ldots, -2, -1, 0, 1, 2, \ldots\}$.
    \item Множество $\mathbb{R}$ вещественных чисел.
\end{itemize}

Множество, не содержащее элементов, называется \textbf{пустым}. \textbf{Обозначение:} $\emptyset$.

Множества как объекты могут быть элементами других множеств. Множество,
элементами которого являются множества, иногда называют \textbf{семейством}.

Совокупность объектов, которая не является множеством, называется \textbf{классом}.

\subsection*{Задание множеств}
Чтобы задать множество, нужно указать, какие элементы ему принадлежат. Это
указание заключают в пару фигурных скобок, оно может иметь одну из следующих основных форм:

\begin{description}
    \item[перечисление элементов:] $M := \{a, b, c, \ldots, z\}$;
    \item[характеристический предикат:] $M := \{x \mid P(x)\}$ ;
    \item[порождающая процедура:] $M := \{x \mid x := f\}$.
\end{description}

При задании множеств перечислением обозначения элементов разделяют за-
пятыми. \textbf{Характеристический предикат} --- это некоторое условие, выраженное
в форме логического утверждения или процедуры, возвращающей логическое
значение.

\paragraph{Примеры задания множеств}
\begin{enumerate}
    \item $M_9 := \{1, 2, 3, 4, 5, 6, 7, 8, 9\}$. \hfill (Перечисление)
    \item $M_9 := \{n \mid n \in \mathbb{N} \land n < 10\}$. \hfill (Характеристический предикат)
    \item $M_9 := \{n \mid n := 0; \text{ for } i \text{ from } 1 \text{ to } 9 \text{ do } n := n + 1; \text{ yield } n \text{ end for}\}$. \hfill (Порождающая процедура)
\end{enumerate}

\subsection*{Парадокс Рассела}
Возможность задания множеств характеристическим предикатом зависит от пре-
диката. Использование некоторых предикатов для этой цели может приводить
к противоречиям.

Рассмотрим множество $Y$ всех множеств, \textbf{не} содержащих себя в качестве элемента:
$$Y := \{X \mid X \notin X\}$$

Если множество $Y$ существует, то возникает вопрос: $Y \in Y$?
\begin{itemize}
    \item \textbf{Пусть} $Y \in Y$, \textbf{тогда} по определению $Y$, должно следовать $Y \notin Y$. (\textbf{Противоречие})
    \item \textbf{Пусть} $Y \notin Y$, \textbf{тогда} по определению $Y$ (как множества всех множеств, не содержащих себя), должно следовать $Y \in Y$. (\textbf{Противоречие})
\end{itemize}
Получается неустранимое логическое противоречие, известное как \textbf{парадокс Рассела}.

\paragraph{Способы избежать парадокса Рассела}
\begin{enumerate}
    \item \textbf{Ограничить характеристические предикаты:} Предикат должен быть вида $P(x) = x \in A \land Q(x)$, где $A$ --- известное, заведомо существующее множество (\textbf{универсум}). Используют обозначение $\{x \in A \mid Q(x)\}$. Для $Y$ универсум не указан, а потому $Y$ множеством не является.
    \item \textbf{Теория типов:} Объекты имеют тип 0, множества элементов типа 0 имеют тип 1, множества элементов типа 0 и 1 --- тип 2 и т. д. $Y$ не имеет типа и множеством не является.
    \item \textbf{Явный запрет принадлежности множества самому себе:} $X \in X$ --- недопустимый предикат. При аксиоматическом построении теории множеств соответствующая аксиома называется \textbf{аксиомой регулярности}.
\end{enumerate}
