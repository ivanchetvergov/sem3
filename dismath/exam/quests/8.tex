\section{Замыкание отношений. Транзитивное и рефлексивное замыкание. Алгоритм Уоршалла. Представление отношений в программах.}

\subsection*{Замыкание отношений}

Пусть $R$ и $R'$ --- отношения на множестве $M$. Отношение $R'$ называется \textbf{замыканием} $R$ относительно свойства $\mathcal{C}$, если:
\begin{enumerate}
    \item $R'$ обладает свойством $\mathcal{C}$ ($\mathcal{C}(R')$).
    \item $R'$ является надмножеством $R$ ($R \subset R'$).
    \item $R'$ является \textbf{наименьшим} таким объектом: если $\mathcal{C}(R'')$ и $R \subset R''$, то $R' \subset R''$.
\end{enumerate}

\subsubsection*{Транзитивное и Рефлексивное замыкание}

Для отношения $R$ на множестве $M$ вводятся объединения его положительных и неотрицательных степеней (композиций):

\paragraph{Транзитивное замыкание ($R^+$)}
\textbf{Транзитивное замыкание} $R^+$ --- это объединение всех \textbf{положительных} степеней $R$:
$$R^+ \stackrel{\text{def}}{=} \bigcup_{i=1}^\infty R^i = R^1 \cup R^2 \cup R^3 \cup \cdots$$
\textbf{Теорема:} $R^+$ является \textbf{наименьшим транзитивным надмножеством} $R$. $R^+$ содержит пару $(a, b)$ тогда и только тогда, когда существует путь из $a$ в $b$ по отношению $R$ (длиной $\ge 1$).

\paragraph{Рефлексивно-транзитивное замыкание ($R^*$)}
\textbf{Рефлексивно-транзитивное замыкание} $R^*$ --- это объединение всех \textbf{неотрицательных} степеней $R$:
$$R^* \stackrel{\text{def}}{=} \bigcup_{i=0}^\infty R^i = R^0 \cup R^1 \cup R^2 \cup \cdots$$
Поскольку $R^0 = I$ (тождественное отношение), то $R^* = R^+ \cup I$.
\textbf{Теорема:} $R^*$ является \textbf{наименьшим рефлексивным и транзитивным надмножеством} $R$. $R^*$ содержит пару $(a, b)$ тогда и только тогда, когда существует путь из $a$ в $b$ по $R$ (длиной $\ge 0$).

\subsection*{Алгоритм Уоршалла}

\textbf{Алгоритм Уоршалла} (Warshall's Algorithm) используется для эффективного вычисления \textbf{транзитивного замыкания} $R^+$ отношения $R$ на множестве $M$, $|M|=n$. Он имеет сложность $O(n^3)$.

Алгоритм работает с булевой матрицей $\mathbf{R}$ отношения.

\paragraph{Алгоритм 1.11 Вычисление транзитивного замыкания}
\textbf{Вход:} Булева матрица отношения $\mathbf{R}: [1..n, 1..n]$.
\textbf{Выход:} Булева матрица транзитивного замыкания $\mathbf{T}$ (т.е., $\mathbf{R}^+$).

\begin{verbatim}
T := R

for k from 1 to n do
    for i from 1 to n do
        for j from 1 to n do
            T[i, j] := T[i, j] OR (T[i, k] AND T[k, j])
        end for
    end for
end for
\end{verbatim}

\textbf{Смысл:} На $k$-й итерации внешний цикл гарантирует, что если есть путь из $i$ в $j$, проходящий только через промежуточные вершины с номерами $\le k$, то $\mathbf{T}[i, j]$ устанавливается в $1$.

\subsection*{Представление отношений в программах}
\paragraph{Булева матрица (Матрица смежности)}
Отношение $R$ на множестве $A = \{a_1, \ldots, a_n\}$ представляется \textbf{булевой матрицей} $\mathbf{R}[n \times n]$, где:
$$\mathbf{R}[i, j] = 1 \iff (a_i, a_j) \in R.$$
Матричное представление эффективно для:
\begin{itemize}
    \item \textbf{Проверки свойств:} Симметричность ($\mathbf{R} = \mathbf{R}^T$), Рефлексивность (диагональ из $1$).
    \item \textbf{Операций:} Композиция $R_1 \circ R_2$ --- булево произведение $\mathbf{R}_1 \cdot \mathbf{R}_2$.
    \item \textbf{Замыкания:} Алгоритм Уоршалла работает непосредственно с этой матрицей.
\end{itemize}
\textbf{ЗАМЕЧАНИЕ:} Для больших разреженных отношений часто используются другие представления, например, \textbf{списки смежности} (см. Графы).