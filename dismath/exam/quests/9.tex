\section{Функциональное отношение. Инъекция, сюръекция, биекция, тотальность. Образы и прообразы. Суперпозиция функций. Представление функций в программах. Примеры.}

\subsection*{Функциональное отношение (Функция)}

\paragraph{Определение}
Функция $f$ из $A$ в $B$ ($f: A \to B$) --- это \textbf{бинарное отношение} $f \subset A \times B$, обладающее свойством \textbf{однозначности} (функциональности):
$$\forall a \in A, b, c \in B: ((a, b) \in f \land (a, c) \in f \implies b = c).$$
То есть, каждому элементу области отправления соответствует не более одного элемента области прибытия.

\paragraph{Терминология}
\begin{itemize}
    \item $A$ --- \textbf{Область отправления} (Домен).
    \item $B$ --- \textbf{Область прибытия} (Кодомен).
    \item \textbf{Область определения} ($\text{Dom } f$) --- множество $a \in A$, для которых $f(a)$ определено.
    \item \textbf{Область значений} ($\text{Im } f$) --- множество $b \in B$, являющихся значениями функции: $\text{Im } f = \{b \in B \mid \exists a \in A (b = f(a))\}$.
\end{itemize}

\paragraph{Тотальность и Частичность}
\begin{itemize}
    \item Функция $f: A \to B$ называется \textbf{тотальной}, если $\text{Dom } f = A$. (Определена для всех элементов $A$).
    \item Функция $f: A \to B$ называется \textbf{частичной}, если $\text{Dom } f \subsetneq A$.
\end{itemize}
\textbf{Преобразование:} Тотальная функция $f: M \to M$ называется \textbf{преобразованием} над $M$.

\subsection*{Инъекция, Сюръекция и Биекция}

Пусть $f: A \to B$ --- тотальная функция.

\begin{description}
    \item[Инъективная (Инъекция):] Если разным аргументам соответствуют разные значения:
    $$f(a_1) = f(a_2) \implies a_1 = a_2.$$
    \item[Сюръективная (Сюръекция):] Если область значений совпадает с областью прибытия (каждый элемент $B$ является образом хотя бы одного элемента $A$):
    $$\forall b \in B (\exists a \in A (b = f(a))).$$
    \item[Биективная (Биекция):] Если функция является одновременно \textbf{инъективной} и \textbf{сюръективной}. Биекция также называется \textbf{взаимно-однозначным соответствием}.
\end{description}

\subsection*{Образы и Прообразы}

Пусть $f: A \to B$ --- функция, $A_1 \subset A$, $B_1 \subset B$.

\begin{itemize}
    \item \textbf{Образ множества $A_1$ ($f(A_1)$):} Множество значений $B$, полученных из элементов $A_1$:
    $$f(A_1) \stackrel{\text{def}}{=} \{b \in B \mid \exists a \in A_1 (b = f(a))\}.$$
    \item \textbf{Прообраз множества $B_1$ ($f^{-1}(B_1)$):} Множество аргументов $A$, отображающихся в $B_1$:
    $$f^{-1}(B_1) \stackrel{\text{def}}{=} \{a \in A \mid \exists b \in B_1 (b = f(a))\}.$$
\end{itemize}
\textbf{Теорема:} Переход к образам ($F: \mathbf{2^A} \to \mathbf{2^B}$) и прообразам ($F^{-1}: \mathbf{2^B} \to \mathbf{2^A}$) также являются функциями.

\subsection*{Суперпозиция функций}

\textbf{Композиция функций} называется \textbf{суперпозицией} и обозначается $\circ$ (как и композиция отношений).

\paragraph{Определение}
Если $f: A \to B$ и $g: B \to C$, то суперпозиция $g \circ f$ --- это функция $g \circ f: A \to C$, определяемая как:
$$(g \circ f)(x) \stackrel{\text{def}}{=} g(f(x)).$$
\textbf{Теорема:} Суперпозиция функций является функцией:
$$f: A \to B \land g: B \to C \implies g \circ f: A \to C.$$

\subsection*{Представление функций в программах}

\paragraph{Массивы (Табулирование)}
Если область отправления $A$ \textbf{конечна и не очень велика}, функция представляется \textbf{массивом} ($\text{array} [A] \text{ of } B$).
\begin{itemize}
    \item Значение функции $f(a)$ вычисляется как $M[a]$ (обращение по индексу).
    \item \textbf{Эффективность:} Вычисление значения происходит за $O(1)$ (константное время).
    \item Функции нескольких аргументов ($f(a_1, \ldots, a_n)$) представляются \textbf{многомерными массивами}.
\end{itemize}

\paragraph{Процедуры (Алгоритмическое представление)}
Если множество $A$ \textbf{велико или бесконечно}, функция представляется \textbf{процедурой} (блоком кода), которая вычисляет значение $b$ по заданному аргументу $a$.
\begin{itemize}
    \item В языках программирования такие процедуры также называются \textbf{функциями} (ключевое слово \texttt{function}).
    \item Свойство функциональности (однозначность) обеспечивается оператором \texttt{return} (возврат единственного значения).
\end{itemize}