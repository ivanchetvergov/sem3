\section{Разбиения и покрытия. Булеан. Свойства операций над множествами. Диаграммы Эйлера-Венна.}

\subsection*{Разбиения и покрытия}

Пусть $\mathcal{E} = \{E_i\}_{i \in I}$ --- некоторое \textbf{семейство подмножеств} множества $M$, $E_i \subset M$.

\paragraph{Покрытие}
Семейство $\mathcal{E}$ называется \textbf{покрытием} множества $M$, если каждый элемент $x \in M$ принадлежит хотя бы одному из множеств $E_i$:
$$\forall x \in M (\exists i \in I (x \in E_i)).$$

\paragraph{Дизъюнктное семейство}
Семейство $\mathcal{E}$ называется \textbf{дизъюнктным}, если элементы этого семейства \textbf{попарно не пересекаются}:
$$\forall i, j \in I (i \ne j \implies E_i \cap E_j = \emptyset).$$

\paragraph{Разбиение}
\textbf{Дизъюнктное покрытие} называется \textbf{разбиением} множества $M$. Элементы разбиения (подмножества $E_i$) часто называют \textbf{блоками} разбиения.

\textbf{Пример:} Пусть $M = \{1, 2, 3\}$.
\begin{itemize}
    \item $\{\{1, 2\}, \{2, 3\}, \{3, 1\}\}$ --- покрытие, но не разбиение (есть пересечения).
    \item $\{\{1\}, \{2\}, \{3\}\}$ --- разбиение (и покрытие).
    \item $\{\{1\}, \{2\}\}$ --- дизъюнктное, но не покрытие, следовательно не разбиение.
\end{itemize}

\subsection*{Булеан (Множество подмножеств)}

\paragraph{Определение Булеана}
\textbf{Множество всех подмножеств} множества $M$ называется \textbf{булеаном} множества $M$ и обозначается $\mathbf{2^M}$ или $\mathcal{P}(M)$:
$$\mathbf{2^M} \stackrel{\text{def}}{=} \{A \mid A \subset M\}.$$

\paragraph{Теорема о мощности Булеана}
Если множество $M$ конечно, то мощность его булеана равна $2$ в степени мощности $M$:
$$| \mathbf{2^M} | = 2^{|M|}.$$

\paragraph{Алгебра подмножеств}
Множество всех подмножеств универсума $U$, снабженное операциями пересечения, объединения, разности и дополнения, образует \textbf{алгебру подмножеств} множества $U$.

\subsection*{Свойства операций над множествами}

Пусть задан универсум $U$. Тогда $\forall A, B, C \subset U$ выполняются следующие равенства (где $\overline{A}$ --- дополнение $A$ до $U$, т.е. $U \setminus A$):

\begin{enumerate}
    \item \textbf{Идемпотентность:}
    $$A \cup A = A, \quad A \cap A = A$$
    \item \textbf{Коммутативность:}
    $$A \cup B = B \cup A, \quad A \cap B = B \cap A$$
    \item \textbf{Ассоциативность:}
    $$A \cup (B \cup C) = (A \cup B) \cup C, \quad A \cap (B \cap C) = (A \cap B) \cap C$$
    \item \textbf{Дистрибутивность:}
    $$A \cup (B \cap C) = (A \cup B) \cap (A \cup C)$$
    $$A \cap (B \cup C) = (A \cap B) \cup (A \cap C)$$
    \item \textbf{Поглощение:}
    $$(A \cap B) \cup A = A, \quad (A \cup B) \cap A = A$$
    \item \textbf{Свойства нуля ($\emptyset$):}
    $$A \cup \emptyset = A, \quad A \cap \emptyset = \emptyset$$
    \item \textbf{Свойства единицы ($U$):}
    $$A \cup U = U, \quad A \cap U = A$$
    \item \textbf{Инволютивность (двойное дополнение):}
    $$\overline{\overline{A}} = A$$
    \item \textbf{Законы де Моргана:}
    $$\overline{A \cap B} = \overline{A} \cup \overline{B}, \quad \overline{A \cup B} = \overline{A} \cap \overline{B}$$
    \item \textbf{Свойства дополнения:}
    $$A \cup \overline{A} = U, \quad A \cap \overline{A} = \emptyset$$
    \item \textbf{Выражение для разности:}
    $$A \setminus B = A \cap \overline{B}$$
\end{enumerate}
