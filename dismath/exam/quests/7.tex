\section{Свойства отношений. Представление отношений, операций над ними и их свойств матрицами. Примеры.}


\subsection*{Свойства бинарных отношений}

Пусть $R$ --- отношение на множестве $A$ ($R \subset A^2$), а $a, b, c \in A$. Отношение $R$ называется:

\begin{description}
    \item[Рефлексивным:] если $\forall a \in A : (a R a)$.
    \item[Антирефлексивным:] если $\forall a \in A : (\neg a R a)$.
    \item[Симметричным:] если $\forall a, b \in A : (a R b \implies b R a)$.
    \item[Антисимметричным:] если $\forall a, b \in A : (a R b \land b R a \implies a = b)$.
    \item[Транзитивным:] если $\forall a, b, c \in A : (a R b \land b R c \implies a R c)$.
    \item[Линейным (Полным):] если $\forall a, b \in A : (a = b \lor a R b \lor b R a)$.
\end{description}

\paragraph{Свойства, выраженные через операции (Теорема)}
Пусть $R \subset A^2$. $I$ --- тождественное отношение на $A$, $U = A^2$ --- универсальное отношение.
\begin{enumerate}
    \item $R$ рефлексивно $\iff I \subset R$.
    \item $R$ симметрично $\iff R = R^{-1}$ (отношение равно своему обратному).
    \item $R$ транзитивно $\iff R \circ R \subset R$ (содержится в самом отношении).
    \item $R$ антисимметрично $\iff R \cap R^{-1} \subset I$.
    \item $R$ антирефлексивно $\iff R \cap I = \emptyset$.
    \item $R$ линейно $\iff R \cup I \cup R^{-1} = U$.
\end{enumerate}

\subsection*{Матричное представление отношений}

Пусть $R$ --- отношение на конечном множестве $A = \{a_1, \ldots, a_n\}$, $|A|=n$. Отношение $R$ представляется \textbf{булевой матрицей} $\mathbf{R}: \text{array} [1..n, 1..n] \text{ of } 0..1$:
$$\mathbf{R}[i, j] = 1 \iff a_i R a_j.$$

\paragraph{Матричное представление свойств}
Для булевой матрицы $\mathbf{R}$ отношения $R$ на $A$:
\begin{itemize}
    \item $R$ \textbf{рефлексивно} $\iff$ Главная диагональ $\mathbf{R}$ состоит из единиц ($\forall i: \mathbf{R}[i, i] = 1$).
    \item $R$ \textbf{антирефлексивно} $\iff$ Главная диагональ $\mathbf{R}$ состоит из нулей ($\forall i: \mathbf{R}[i, i] = 0$).
    \item $R$ \textbf{симметрично} $\iff \mathbf{R} = \mathbf{R}^T$ (матрица симметрична).
    \item $R$ \textbf{антисимметрично} $\iff$ Для $i \ne j$: $\mathbf{R}[i, j] = 1 \implies \mathbf{R}[j, i] = 0$.
    \item $R$ \textbf{транзитивно} $\iff \mathbf{R} \cdot \mathbf{R} \subset \mathbf{R}$ (где $\cdot$ --- булево умножение матриц).
\end{itemize}

\paragraph{Операции над отношениями через матрицы (Теоремы)}
Пусть $\mathbf{R}_1$ и $\mathbf{R}_2$ --- булевы матрицы отношений $R_1$ и $R_2$.

\begin{enumerate}
    \item \textbf{Обратное отношение ($R^{-1}$):} Матрица обратного отношения равна \textbf{транспонированной} матрице:
    $$\mathbf{R^{-1}} = \mathbf{R}^T.$$

    \item \textbf{Дополнение отношения ($\overline{R}$):} Матрица дополнения получается \textbf{инвертированием} всех элементов (замена $1 \to 0$, $0 \to 1$):
    $$\overline{\mathbf{R}}[i, j] = 1 - \mathbf{R}[i, j].$$
    
    \item \textbf{Объединение ($R_1 \cup R_2$):} Матрица объединения равна \textbf{булевой дизъюнкции} (логическое $\lor$) матриц:
    $$\mathbf{R_1 \cup R_2} = \mathbf{R}_1 \lor \mathbf{R}_2.$$

    \item \textbf{Пересечение ($R_1 \cap R_2$):} Матрица пересечения равна \textbf{булевой конъюнкции} (логическое $\land$) матриц:
    $$\mathbf{R_1 \cap R_2} = \mathbf{R}_1 \land \mathbf{R}_2.$$

    \item \textbf{Композиция ($R_1 \circ R_2$):} Матрица композиции равна \textbf{булевому произведению} матриц:
    $$\mathbf{R_1 \circ R_2} = \mathbf{R}_1 \cdot \mathbf{R}_2.$$
    (Булево произведение: $(\mathbf{R}_1 \cdot \mathbf{R}_2)[i, j] = \bigvee_{k=1}^n (\mathbf{R}_1[i, k] \land \mathbf{R}_2[k, j])$).
\end{enumerate}