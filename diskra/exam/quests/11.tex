\section{Отношение порядка. Минимальные элементы. Верхние и нижние границы. Монотонность. Вполне упорядоченные множества. Диаграммы Хассе. Примеры.}


\subsection*{Отношение порядка}

\paragraph{Определение}
Отношение $\prec$ на множестве $M$ называется \textbf{отношением порядка}, если оно одновременно:
\begin{enumerate}
    \item \textbf{Антисимметрично:} $\forall a, b \in M (a \prec b \land b \prec a \implies a = b)$.
    \item \textbf{Транзитивно:} $\forall a, b, c \in M (a \prec b \land b \prec c \implies a \prec c)$.
\end{enumerate}

\paragraph{Типы порядка}
\begin{itemize}
    \item \textbf{Нестрогий порядок} ($\preceq$): Отношение порядка, которое \textbf{рефлексивно} ($\forall a: a \preceq a$).
    \item \textbf{Строгий порядок} ($\prec$): Отношение порядка, которое \textbf{антирефлексивно} ($\forall a: \neg a \prec a$).
    \item \textbf{Линейный порядок} (Полный): Отношение порядка, которое \textbf{линейно} ($\forall a, b: a = b \lor a \prec b \lor b \prec a$).
    \item \textbf{Частичный порядок:} Отношение порядка, которое \textbf{не является линейным}.
\end{itemize}

\subsection*{Границы и специальные элементы}

\paragraph{Минимальные и максимальные элементы}
*(В представленном фрагменте нет явного определения минимальных элементов, но они подразумеваются в разделе о вполне упорядоченных множествах.)*

Пусть $M$ --- частично упорядоченное множество.
\begin{itemize}
    \item Элемент $m \in M$ называется \textbf{минимальным}, если $\neg \exists x \in M (x \prec m)$.
    \item Элемент $m \in M$ называется \textbf{максимальным}, если $\neg \exists x \in M (m \prec x)$.
\end{itemize}

\paragraph{Верхние и нижние границы}
Пусть $X \subset M$ --- подмножество упорядоченного множества $M$.
\begin{itemize}
    \item Элемент $m \in M$ --- \textbf{Нижняя граница} $X$, если $\forall x \in X (m \preceq x)$.
    \item Элемент $m \in M$ --- \textbf{Верхняя граница} $X$, если $\forall x \in X (x \preceq m)$.
\end{itemize}
\textbf{Инфимум (inf $X$):} Наибольшая нижняя граница множества $X$.
\textbf{Супремум (sup $X$):} Наименьшая верхняя граница множества $X$.

\subsection*{Монотонные функции}

Пусть $A$ и $B$ --- упорядоченные множества с отношением $\preceq$, и $f: A \to B$.

\begin{description}
    \item[Монотонно возрастающая:] Если $a_1 \preceq a_2 \implies f(a_1) \preceq f(a_2)$.
    \item[Строго монотонно возрастающая:] Если $a_1 \prec a_2 \implies f(a_1) \prec f(a_2)$.
    \item[Монотонно убывающая:] Если $a_1 \preceq a_2 \implies f(a_2) \preceq f(a_1)$.
    \item[Строго монотонно убывающая:] Если $a_1 \prec a_2 \implies f(a_2) \prec f(a_1)$.
\end{description}
\textbf{Монотонные функции} --- это монотонно возрастающие или убывающие функции.

\subsection*{Вполне упорядоченные множества}

\paragraph{Определение}
Частично упорядоченное множество $X$ называется \textbf{вполне упорядоченным}, если \textbf{любое его непустое подмножество} имеет \textbf{минимальный элемент}.

\textbf{Следствие:} Вполне упорядоченное множество всегда \textbf{линейно упорядочено}, поскольку для любых двух элементов $a, b \in X$, подмножество $\{a, b\}$ должно иметь минимальный элемент, что означает, что $a \preceq b$ или $b \preceq a$.

\subsection*{Диаграммы Хассе}

\textbf{Диаграмма Хассе} --- это графическое представление конечного \textbf{частично упорядоченного множества} ($M, \preceq$), где:
\begin{enumerate}
    \item Элементы $M$ представлены узлами.
    \item Элементы, связанные отношением рефлексивности и транзитивности, \textbf{не отображаются} (транзитивное сокращение).
    \item Если $a \prec b$ и нет $c$ такого, что $a \prec c \prec b$, то $b$ располагается выше $a$ и соединяется с ним линией.
\end{enumerate}
Диаграмма Хассе наглядно показывает \textbf{непосредственное следование} элементов в порядке.