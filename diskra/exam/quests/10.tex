\section{Отношение эквивалентности. Классы эквивалентности. Фактор-множества. Ядро функционального отношения и множества уровня.}


\subsection*{Отношение эквивалентности}

\paragraph{Определение}
Отношение $\approx \subset M^2$ на множестве $M$ называется \textbf{отношением эквивалентности}, если оно одновременно:
\begin{enumerate}
    \item \textbf{Рефлексивно:} $\forall x \in M (x \approx x)$.
    \item \textbf{Симметрично:} $\forall x, y \in M (x \approx y \implies y \approx x)$.
    \item \textbf{Транзитивно:} $\forall x, y, z \in M (x \approx y \land y \approx z \implies x \approx z)$.
\end{enumerate}

\paragraph{Примеры}
\begin{itemize}
    \item Равенство чисел ($\mathbb{R}^2$), равенство множеств ($\mathbf{2^M}^2$).
    \item Равномощность множеств ($\mathbf{2^M}^2$).
\end{itemize}

\subsection*{Классы эквивалентности и Фактор-множество}

\paragraph{Класс эквивалентности}
Пусть $\approx$ --- отношение эквивалентности на $M$. \textbf{Классом эквивалентности} для элемента $x \in M$ называется подмножество элементов $M$, эквивалентных $x$:
$$[x]_\approx \stackrel{\text{def}}{=} \{y \in M \mid y \approx x\}.$$

\paragraph{Свойства классов эквивалентности}
\begin{itemize}
    \item \textbf{Непустота:} $\forall a \in M ([a] \ne \emptyset)$. (Следует из рефлексивности).
    \item \textbf{Равенство:} $a \approx b \implies [a] = [b]$.
    \item \textbf{Дизъюнктность:} $a \not\approx b \implies [a] \cap [b] = \emptyset$.
\end{itemize}

\paragraph{Теорема о разбиении}
Если $\approx$ --- отношение эквивалентности на $M$, то классы эквивалентности $[x]_\approx$ образуют \textbf{разбиение} множества $M$. И обратно, всякое разбиение множества $M$ определяет отношение эквивалентности, классами которого являются блоки разбиения.

\paragraph{Фактор-множество}
Если $R$ --- отношение эквивалентности на $M$, то \textbf{фактор-множеством} $M$ относительно $R$ называется множество всех классов эквивалентности по $R$:
$$M/R \stackrel{\text{def}}{=} \{[x]_R \mid x \in M\}.$$
Фактор-множество является подмножеством булеана: $M/R \subset \mathbf{2^M}$.

\paragraph{Отождествление}
Функция $\text{nat}_R: M \to M/R$, определяемая как $\text{nat}_R(x) \stackrel{\text{def}}{=} [x]_R$, называется \textbf{отождествлением} (естественной сюръекцией).

\subsection*{Ядро функционального отношения}

\paragraph{Ядро функционального отношения}
Всякое функциональное отношение (функция) $f: A \to B$ имеет \textbf{ядро}, которое является отношением на области определения $\text{Dom } f$:
$$\ker f \stackrel{\text{def}}{=} f \circ f^{-1} \subset (\text{Dom } f)^2.$$
Ядро $\ker f$ связывает два элемента $a_1, a_2 \in A$ тогда и только тогда, когда они имеют \textbf{одинаковые значения} при отображении $f$:
$$a_1 \ker f a_2 \iff f(a_1) = f(a_2).$$

\paragraph{Теорема}
Ядро функционального отношения $\ker f$ является \textbf{отношением эквивалентности} на его области определения $\text{Dom } f$.

\paragraph{Множества уровня}
Множества, на которые разбивает $\text{Dom } f$ ядро $\ker f$, являются классами эквивалентности $\ker f$. Эти классы называются \textbf{множествами уровня} (или \textbf{слоями}) функции $f$.
$$\text{Dom } f / \ker f = \{[a]_{\ker f} \mid a \in \text{Dom } f\}.$$
Каждый класс $[a]_{\ker f}$ состоит из всех элементов, которые отображаются функцией $f$ в одно и то же значение $f(a)$.
$$[a]_{\ker f} = \{x \in \text{Dom } f \mid f(x) = f(a)\}.$$
Множества уровня функции $f$ образуют разбиение области определения $\text{Dom } f$.