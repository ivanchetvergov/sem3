\section{Виды морфизмов. Гомоморфизм, изоморфизм. Примеры.}


\subsection*{Гомоморфизм}

\paragraph{Определение}
Пусть $\mathcal{A} = (A; \Sigma_\mathcal{A})$ и $\mathcal{B} = (B; \Sigma_\mathcal{B})$ --- две алгебры \textbf{одного типа} (т.е., сигнатуры имеют одинаковое количество операций с одинаковыми арностями).
Функция $f: A \to B$ называется \textbf{гомоморфизмом} из $\mathcal{A}$ в $\mathcal{B}$, если она \textbf{согласована с операциями} $\varphi_i$ и $\psi_i$ (где $\psi_i$ соответствует $\varphi_i$):
$$\forall i \in \{1, \ldots, m\}: f(\varphi_i(a_1, \ldots, a_{n_i})) = \psi_i(f(a_1), \ldots, f(a_{n_i})).$$
Образно говоря, гомоморфизм «уважает» операции: результат операции над аргументами в $A$ и последующее отображение в $B$ равносильно отображению аргументов в $B$ и последующей операции в $B$.

\paragraph{Коммутативная диаграмма}
Условие гомоморфизма для одной бинарной операции $\varphi$ может быть записано через суперпозицию:
$$f \circ \varphi_\mathcal{A} = \varphi_\mathcal{B} \circ (f \times f).$$

\paragraph{Виды гомоморфизмов}
Термин \textbf{морфизм} используется как собирательное понятие для всех следующих видов отображений:
\begin{itemize}
    \item \textbf{Эндоморфизм:} Гомоморфизм $f: \mathcal{A} \to \mathcal{A}$ (отображение алгебры на саму себя).
    \item \textbf{Мономорфизм:} Гомоморфизм, который является \textbf{инъекцией}.
    \item \textbf{Эпиморфизм:} Гомоморфизм, который является \textbf{сюръекцией}.
\end{itemize}

\paragraph{Пример Гомоморфизма}
Пусть $\mathcal{A} = (\mathbb{N}; +)$ и $\mathcal{B} = (\mathbb{N}_{10}; +_{10})$ (сложение по модулю 10).
Функция $f(a) = a \bmod 10$ является гомоморфизмом из $\mathcal{A}$ в $\mathcal{B}$.
$$f(a_1 + a_2) = (a_1 + a_2) \bmod 10$$
$$f(a_1) +_{10} f(a_2) = (a_1 \bmod 10 + a_2 \bmod 10) \bmod 10$$
Обе стороны равны, что доказывает гомоморфизм.

\subsection*{Изоморфизм}

\paragraph{Определение}
\textbf{Изоморфизм} $f: \mathcal{A} \to \mathcal{B}$ --- это гомоморфизм, который является \textbf{биекцией} (одновременно инъекцией и сюръекцией).
Если между алгебрами $\mathcal{A}$ и $\mathcal{B}$ существует изоморфизм, они называются \textbf{изоморфными} ($\mathcal{A} \cong \mathcal{B}$).

\paragraph{Свойства Изоморфизма}
\begin{itemize}
    \item \textbf{Обратный изоморфизм:} Если $f: \mathcal{A} \to \mathcal{B}$ --- изоморфизм, то обратная функция $f^{-1}: \mathcal{B} \to \mathcal{A}$ также является изоморфизмом.
    \item \textbf{Эквивалентность:} Отношение изоморфизма ($\cong$) на множестве однотипных алгебр является \textbf{отношением эквивалентности}.
    \item \textbf{Автоморфизм:} Изоморфизм $f: \mathcal{A} \to \mathcal{A}$ называется \textbf{автоморфизмом}.
\end{itemize}

\paragraph{Значение Изоморфизма}
Изоморфные алгебры имеют \textbf{одинаковую структуру} с точки зрения алгебраических свойств. Любое свойство, выраженное в сигнатуре $\Sigma$, верное в $\mathcal{A}$, автоматически верно и в $\mathcal{B}$. Это позволяет изучать алгебраические структуры с \textbf{точностью до изоморфизма}.

\paragraph{Примеры Изоморфизма}
\begin{enumerate}
    \item Алгебра положительных вещественных чисел по умножению изоморфна алгебре всех вещественных чисел по сложению:
    $$\mathcal{A} = (\mathbb{R}^+; \cdot) \cong \mathcal{B} = (\mathbb{R}; +).$$
    Изоморфизм задается функцией $f(x) = \ln(x)$.
    \item Алгебра натуральных чисел по сложению изоморфна алгебре четных натуральных чисел по сложению:
    $$\mathcal{A} = (\mathbb{N}; +) \cong \mathcal{B} = (\{2k \mid k \in \mathbb{N}\}; +).$$
    Изоморфизм задается функцией $f(k) = 2k$.
\end{enumerate}