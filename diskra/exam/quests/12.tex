\section{Характеристические функции множеств и отношений. Функции максимума и минимума.}

\subsection*{Характеристическая функция множества}
Я ХУЙ ЗНАЕТ НОРМ НЕ НОРМ. Я НЕ НАШЕЛ (найдете киньте ISSUE)
\paragraph{Определение}
Пусть $U$ --- универсум (конечное или бесконечное множество), и $A$ --- любое его подмножество ($A \subset U$). \textbf{Характеристическая функция} $\chi_A$ множества $A$ --- это тотальная функция $\chi_A: U \to \{0, 1\}$, которая принимает значение $1$, если элемент принадлежит $A$, и $0$, если не принадлежит:
$$\chi_A(x) \stackrel{\text{def}}{=} \begin{cases}
1, & \text{если } x \in A \\
0, & \text{если } x \notin A
\end{cases}$$
Характеристическая функция устанавливает взаимно-однозначное соответствие между элементами булеана $\mathbf{2^U}$ и множеством функций $U \to \{0, 1\}$.

\paragraph{Характеристическая функция операций над множествами}
Пусть $A$ и $B$ --- подмножества $U$. Операции над множествами выражаются через логические операции над их характеристическими функциями:

\begin{itemize}
    \item \textbf{Дополнение} ($\overline{A}$): $\chi_{\overline{A}}(x) = 1 - \chi_A(x)$.
    \item \textbf{Пересечение} ($A \cap B$): $\chi_{A \cap B}(x) = \chi_A(x) \land \chi_B(x) = \chi_A(x) \cdot \chi_B(x)$.
    \item \textbf{Объединение} ($A \cup B$): $\chi_{A \cup B}(x) = \chi_A(x) \lor \chi_B(x) = \chi_A(x) + \chi_B(x) - \chi_A(x)\cdot\chi_B(x)$.
    \item \textbf{Симметрическая разность} ($A \Delta B$): $\chi_{A \Delta B}(x) = \chi_A(x) \oplus \chi_B(x)$ ($\oplus$ --- исключающее "ИЛИ").
\end{itemize}

\subsection*{Характеристическая функция бинарного отношения}

\paragraph{Определение}
Пусть $A$ и $B$ --- множества, и $R$ --- бинарное отношение между ними ($R \subset A \times B$). \textbf{Характеристическая функция отношения} $\chi_R$ --- это тотальная функция $\chi_R: A \times B \to \{0, 1\}$, которая определена на декартовом произведении и принимает значение $1$, если пара принадлежит $R$, и $0$, если не принадлежит:
$$\chi_R(a, b) \stackrel{\text{def}}{=} \begin{cases}
1, & \text{если } (a, b) \in R \quad (\text{т.е., } a R b) \\
0, & \text{если } (a, b) \notin R
\end{cases}$$
Для конечных множеств $A$ и $B$ характеристическая функция отношения эквивалентна его \textbf{булевой матрице} (см. Билет 7), где $\mathbf{R}[i, j] = \chi_R(a_i, b_j)$.

\subsection*{Функции максимума и минимума}

Функции максимума ($\max$) и минимума ($\min$) используются для определения кратностей элементов при теоретико-множественных операциях над \textbf{мультимножествами} (см. Билет 5).

\paragraph{Функция максимума (для мультимножеств)}
Пусть $\mathcal{A}$ и $\mathcal{B}$ --- мультимножества, и $\alpha_x, \beta_x$ --- кратности элемента $x$ в $\mathcal{A}$ и $\mathcal{B}$ соответственно.
\textbf{Кратность элемента $x$ в $\mathcal{A} \cup \mathcal{B}$:}
$$\text{count}_{\mathcal{A} \cup \mathcal{B}}(x) = \max(\alpha_x, \beta_x).$$
\textbf{Определение $\max$:} Для двух элементов $a, b$ в упорядоченном множестве $M$ (например, $\mathbb{N}$ или $\mathbb{R}$):
$$\max(a, b) \stackrel{\text{def}}{=} \begin{cases}
a, & \text{если } a \ge b \\
b, & \text{если } a < b
\end{cases}$$

\paragraph{Функция минимума (для мультимножеств)}
\textbf{Кратность элемента $x$ в $\mathcal{A} \cap \mathcal{B}$:}
$$\text{count}_{\mathcal{A} \cap \mathcal{B}}(x) = \min(\alpha_x, \beta_x).$$
\textbf{Определение $\min$:} Для двух элементов $a, b$ в упорядоченном множестве $M$:
$$\min(a, b) \stackrel{\text{def}}{=} \begin{cases}
a, & \text{если } a \le b \\
b, & \text{если } a > b
\end{cases}$$

\textbf{ЗАМЕЧАНИЕ:} Функции $\max$ и $\min$ также используются для определения \textbf{верхних} (супремум, $\sup$) и \textbf{нижних} (инфимум, $\inf$) границ в частично упорядоченных множествах (см. Билет 11).