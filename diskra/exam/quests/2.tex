\section{Алгебра подмножеств. Сравнение множеств. Мощность множества. Операции над множествами.}

\subsection*{Сравнение множеств}

Для конструирования новых множеств из имеющихся определяются \textbf{операции над множествами}.

\paragraph{Подмножество и Надмножество}
Множество $A$ \textbf{содержится} в множестве $B$ (или $B$ \textbf{включает} $A$), если каждый элемент множества $A$ есть элемент множества $B$:
$$A \subset B \stackrel{\text{def}}{\iff} \forall x (x \in A \implies x \in B).$$
В этом случае $A$ называется \textbf{подмножеством} $B$, $B$ --- \textbf{надмножеством} $A$.
По определению, $\forall M (\emptyset \subset M)$.

\paragraph{Равенство множеств}
Два множества \textbf{равны}, если они являются подмножествами друг друга:
$$A = B \iff (A \subset B \land B \subset A).$$

\paragraph{Свойства включения множеств (Теорема)}
Включение множеств обладает следующими свойствами:
\begin{enumerate}
    \item Рефлексивность: $\forall A (A \subset A)$.
    \item Антисимметричность: $\forall A, B (A \subset B \land B \subset A \implies A = B)$.
    \item Транзитивность: $\forall A, B, C (A \subset B \land B \subset C \implies A \subset C)$.
\end{enumerate}

\subsection*{Мощность множества}

\paragraph{Равномощные множества}
Говорят, что между множествами $A$ и $B$ установлено \textbf{взаимно-однозначное соответствие} (или они \textbf{изоморфны}, $A \sim B$), если:
\begin{itemize}
    \item Каждому элементу $a \in A$ соответствует один и только один элемент $b \in B$.
    \item Для каждого элемента $b \in B$ соответствует один и только один элемент $a \in A$.
\end{itemize}
Если $a \in A$ соответствует $b \in B$, обозначают $a \mapsto b$.

\textbf{Пример:} Соответствие $n \mapsto 2n$ устанавливает взаимно-однозначное соответствие между множеством натуральных чисел $\mathbb{N}$ и множеством чётных натуральных чисел $2\mathbb{N}$. $(\mathbb{N} \sim 2\mathbb{N})$

\paragraph{Одинаковая мощность}
Если между двумя множествами $A$ и $B$ может быть установлено взаимно-однозначное соответствие, то говорят, что множества имеют \textbf{одинаковую мощность} или \textbf{равномощны}, и записывают это как $|A| = |B|$.
$$|A| = |B| \iff A \sim B.$$

\paragraph{Свойства равномощности (Теорема)}
Равномощность множеств обладает следующими свойствами:
\begin{enumerate}
    \item Рефлексивность: $\forall A (|A| = |A|)$.
    \item Симметричность: $\forall A, B (|A| = |B| \implies |B| = |A|)$.
    \item Транзитивность: $\forall A, B, C (|A| = |B| \land |B| = |C| \implies |A| = |C|)$.
\end{enumerate}

\subsection*{Конечные и бесконечные множества}

\paragraph{Конечное множество}
Множество $A$ называется \textbf{конечным}, если у него нет равномощного \textbf{собственного} подмножества ($B \subset A$ и $B \ne A$):
$$\forall B ((B \subset A \land |B| = |A|) \implies B = A).$$
Обозначение: $|A| < \infty$.

\paragraph{Бесконечное множество}
Множество $A$ называется \textbf{бесконечным}, если оно равномощно некоторому своему собственному подмножеству:
$$\exists B (B \subset A \land |B| = |A| \land B \ne A).$$
Обозначение: $|A| = \infty$.

\textbf{Пример:} Множество $\mathbb{N}$ бесконечно, $|\mathbb{N}| = \infty$, так как оно равномощно своему собственному подмножеству чётных чисел $2\mathbb{N}$.

\textbf{Теорема:} Множество, имеющее бесконечное подмножество, бесконечно:
$$(B \subset A \land |B| = \infty) \implies (|A| = \infty).$$

\paragraph{Мощность конечного множества}
\textbf{Теорема:} Любое непустое конечное множество равномощно некоторому отрезку натурального ряда:
$$\forall A (A \ne \emptyset \land |A| < \infty \implies \exists k \in \mathbb{N} (|A| = |\{1, \ldots, k\}|)).$$

\subsection*{Операции над множествами}

Обычно рассматриваются следующие операции над множествами:

\begin{description}
    \item[Объединение (Union):] $A \cup B \stackrel{\text{def}}{=} \{x \mid x \in A \lor x \in B\}$
    \item[Пересечение (Intersection):] $A \cap B \stackrel{\text{def}}{=} \{x \mid x \in A \land x \in B\}$
    \item[Разность (Difference):] $A \setminus B \stackrel{\text{def}}{=} \{x \mid x \in A \land x \notin B\}$
    \item[Симметрическая разность (Symmetric Difference):]  $A \Delta B \stackrel{\text{def}}{=} (A \cup B) \setminus (A \cap B) = \{x \mid (x \in A \land x \notin B) \lor (x \notin A \land x \in B)\}$
    \item[Дополнение (Complement):] $\overline{A} \stackrel{\text{def}}{=} \{x \mid x \notin A\}$
\end{description}

\textbf{ЗАМЕЧАНИЕ:} Операция дополнения $\overline{A}$ определена только при заданном \textbf{универсуме} $U$: $\overline{A} = U \setminus A$.

\paragraph{Формулы для мощностей (для конечных множеств)}
\begin{itemize}
    \item $|\text{A} \cup \text{B}| = |\text{A}| + |\text{B}| - |\text{A} \cap \text{B}|$ 
    \item $|\text{A} \setminus \text{B}| = |\text{A}| - |\text{A} \cap \text{B}|$
    \item $|\text{A} \Delta \text{B}| = |\text{A}| + |\text{B}| - 2|\text{A} \cap \text{B}|$
\end{itemize}